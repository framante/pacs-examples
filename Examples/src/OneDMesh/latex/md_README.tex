\#\+Classes to handle a simple mesh\#

You need first to compile and install R\+K45\+:

go to {\ttfamily Examples/src/\+R\+K45} and do \begin{DoxyVerb}make distclean
make dynamic DEBUG=no  (to disable debugging)
make install
make distclean
\end{DoxyVerb}


Include files go to the {\ttfamily Examples/include} and the library {\ttfamily librk45.\+so} in {\ttfamily Examples/lib}.

Go back to this directory and run make distclean; make dynamic (or static)

Make sure to set {\ttfamily L\+D\+\_\+\+L\+I\+B\+R\+A\+R\+Y\+\_\+\+P\+A\+TH} accordingly (not necessary anymore). In

{\ttfamily run\+Test\+Generator.\+sh} you have a scrit that runs an example of how to launch a test.

{\bfseries{IF Y\+OU W\+A\+NT TO I\+N\+S\+T\+A\+LL T\+HE C\+O\+DE (needed for other examples) do}} \begin{DoxyVerb}make distclean
make dynamic DEBUG=no  (to disable debugging)
make install
make distclean
\end{DoxyVerb}


This software provides a class to store a simple 1D mesh of a one dimensional domain represented by an interval. It uses a policy, that derives from {\ttfamily One\+D\+Mesh\+Generator}, to implement the actual mesh generation process. We provide two policies\+: one for uniform mesh generation, the other that allows to prescribe a variable mesh spacing. The latter needs to integrate a ordinary differential equation to locate the mesh nodes; to this purpose it uses one of the runge kutta adaptive schemes provided in the directory {\ttfamily rk45/}

{\ttfamily main\+\_\+\+Test\+Generator} is a simple application to test the sofware and {\ttfamily run\+\_\+test\+\_\+generator.\+sh} is a script that runs the code and lauches {\ttfamily gnuplot}.

\#\+What do I learn here?\#
\begin{DoxyItemize}
\item An example of the {\itshape strategy design pattern}, or policy. The way the mesh is generated is a policy of the class Mesh1d. According to the policy the mesh can be univorm or with a prescribed spacing;
\item A way to compute the nodes of a mesh by integrating a spacing function using an ode solver. 
\end{DoxyItemize}